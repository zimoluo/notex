\documentclass[12pt]{notex}
\author{User}
\date{\today}
\lecture{Hello World!}

\begin{document}
\maketitle
\tableofcontents
\newpage

\section{This is a Chapter}

This is a chapter.

\section*{This is a Numberless Section}

\subsection{This is a Module}

This is a module.

\subsection*{This is a numberless subsection}

\subsubsection{This is a Topic}

This is a topic.

\subsubsection*{This is a numberless subsbusection}

Try \NoTeX and \RossTeX and more!

\section{Environments}

\begin{theorem}[This is a Theorem]
    \lipsum[1]
\end{theorem}

\begin{proof}
    \lipsum[2]
\end{proof}

\begin{lemma}
    \lipsum[3]
\end{lemma}

\begin{corollary}
    \lipsum[4]
\end{corollary}

\begin{proposition}
    \lipsum[5]
\end{proposition}

\begin{statement*} % One asterisk means no counter is applied
    \lipsum[6]
\end{statement*}

\begin{conjecture*}
    \lipsum[7]
\end{conjecture*}

\begin{definition**} % Two asterisks means this environment uses a different set of counter that displays as '1' rather than '1.1'
    \lipsum[8]
\end{definition**}

\begin{problem**}
    \lipsum[9]
\end{problem**}

\begin{question}
    \lipsum[10]
\end{question}

\begin{claim}
    \lipsum[11]
\end{claim}

\begin{fact}
    \lipsum[12]
\end{fact}

\begin{remark}
    \lipsum[13]
\end{remark}

\begin{exercise}
    \lipsum[14]
\end{exercise}

\begin{solution}
    \lipsum[15]
\end{solution}

\begin{example}
    \lipsum[16]
\end{example}

\appendix

\section{This is an Appendix}

This is an appendix.

\end{document}